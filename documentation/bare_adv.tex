
\documentclass[12pt,journal,compsoc]{IEEEtran}
% The Computer Society requires 12pt.
% If IEEEtran.cls has not been installed into the LaTeX system files,
% manually specify the path to it like:
% \documentclass[10pt,journal,compsoc]{../sty/IEEEtran}

% For Computer Society journals, IEEEtran defaults to the use of 
% Palatino/Palladio as is done in IEEE Computer Society journals.
% To go back to Times Roman, you can use this code:
%\renewcommand{\rmdefault}{ptm}\selectfont

% Some very useful LaTeX packages include:

\usepackage[utf8]{inputenc}
\usepackage{listings}
\usepackage{xcolor}
\usepackage{graphicx}

\ifCLASSOPTIONcompsoc
  \usepackage[caption=false,font=normalsize,labelfont=sf,textfont=sf]{subfig}
\else
  \usepackage[caption=false,font=footnotesize]{subfig}
\fi
% subfig.sty, written by Steven Douglas Cochran, is the modern replacement
% for subfigure.sty, the latter of which is no longer maintained and is
% incompatible with some LaTeX packages including fixltx2e. However,
% subfig.sty requires and automatically loads Axel Sommerfeldt's caption.sty
% which will override IEEEtran.cls' handling of captions and this will result
% in non-IEEE style figure/table captions. To prevent this problem, be sure
% and invoke subfig.sty's "caption=false" package option (available since
% subfig.sty version 1.3, 2005/06/28) as this is will preserve IEEEtran.cls
% handling of captions.
% Note that the Computer Society format requires a larger sans serif font
% than the serif footnote size font used in traditional IEEE formatting
% and thus the need to invoke different subfig.sty package options depending
% on whether compsoc mode has been enabled.
%
% The latest version and documentation of subfig.sty can be obtained at:
% http://www.ctan.org/tex-archive/macros/latex/contrib/subfig/


\include{code_formatting}


% *** Do not adjust lengths that control margins, column widths, etc. ***
% *** Do not use packages that alter fonts (such as pslatex).         ***
% There should be no need to do such things with IEEEtran.cls V1.6 and later.
% (Unless specifically asked to do so by the journal or conference you plan
% to submit to, of course. )


% correct bad hyphenation here
\hyphenation{op-tical net-works semi-conduc-tor}


\begin{document}

%
% paper title
% can use linebreaks \\ within to get better formatting as desired
% Do not put math or special symbols in the title.
\title{Smart Bierdeckel}
%
%
% author names and IEEE memberships
% note positions of commas and nonbreaking spaces ( ~ ) LaTeX will not break
% a structure at a ~ so this keeps an author's name from being broken across
% two lines.
% use \thanks{} to gain access to the first footnote area
% a separate \thanks must be used for each paragraph as LaTeX2e's \thanks
% was not built to handle multiple paragraphs

\author{Achim~Herrmann,
        Achim~Däubler}

% The paper headers
\markboth{Smart Bierdeckel Dokumentation, Februar~2016}%
{%Shell \MakeLowercase{\textit{et al.}}: Bare Advanced Demo of IEEEtran.cls for Journals}
}
% The only time the second header will appear is for the odd numbered pages
% after the title page when using the twoside option.
% 
% *** Note that you probably will NOT want to include the author's ***
% *** name in the headers of peer review papers.                   ***
% You can use \ifCLASSOPTIONpeerreview for conditional compilation here if
% you desire.


% for Computer Society papers, we must declare the abstract and index terms
% PRIOR to the title within the \IEEEtitleabstractindextext IEEEtran
% command as these need to go into the title area created by \maketitle.
% As a general rule, do not put math, special symbols or citations
% in the abstract or keywords.
\IEEEtitleabstractindextext{%
\begin{abstract}
Ziel des Projektes Smart Bierdeckel ist es einen Bierdeckel zu konstruieren, der mitzählt wie oft man sich ein neues Bier geholt hat.
Er besteht aus einer Wägezelle, mit der das Gewicht des darauf abgestellten Getränks gemessen wird.
Außerdem enthält er einen Mikrocontroller, der die gemessenen Gewichtsdaten erhält und daraus errechnet ob ein neues Getränk abgestellt wurde oder nicht.
Wie viele neue Getränke schon abgestellt wurden wird durch LEDs am Bierdeckel veranschaulicht.
\end{abstract}
}


% make the title area
\maketitle


\section{Einleitung}

% The very first letter is a 2 line initial drop letter followed
% by the rest of the first word in caps (small caps for compsoc).
% 
% form to use if the first word consists of a single letter:
% \IEEEPARstart{A}{demo} file is ....
% 
% form to use if you need the single drop letter followed by
% normal text (unknown if ever used by IEEE):
% \IEEEPARstart{A}{}demo file is ....
% 
% Some journals put the first two words in caps:
% \IEEEPARstart{T}{his demo} file is ....
% 
% Here we have the typical use of a "T" for an initial drop letter
% and "HIS" in caps to complete the first word.
\IEEEPARstart{Z}{uerst} wird der Aufbau des Bierdeckels beschrieben.
Es wird erklärt, welche Bestandteile benötigt werden und wie sie funktionieren.
Danach zeigen wir Code, der veraunschaulicht, wie die Gewichtsmessung funktioniert.


\section{Bestandteile}

Unser Bierdeckel besteht aus einer Billigwaage, die die Wägezelle enthält.
Als Mikrocontroller verwenden wir einen Atmega328p. Da dieser auch auf dem Arduino Uno verbaut
ist, konnten wir unsere Andwendung zuert mithilfe des Arduinos auf einem Steckbrett testen.
Um die Gewichtsdaten, die die Wägezelle liefert verarbeiten zu können verwenden wir den HX711,
einen 24 Bit Analog zu Digital Converter für Brückensensoren. Diesen gibt es als fertiges Modul
zu kaufen (siehe Fig. \ref{fig_HX711_board}). Um anzeigen wie viele neue Biere abgestellt verwenden wir 8 LEDs
(4 gelbe und 4 rote, mit 3 mm Durchmesser). Damit nicht alle LEDs einzeln an einen Widerstand
und dann ans Board gelötet werden müssen, haben wir die LEDs in zwei Reihen, mit den
entsprechenden Anschlüssen und Vorwiderständen, auf eine extra Platine gelötet (siehe Fig. \ref{fig_led_platine}).
Die Betriebsspannung wird von 3 AAA Batterien bereitgestellt. Diese liefern zusammen 4,5 Volt.
Das ist ausreichend da der Atmega ab 3,7 Volt ein HIGH liest. Die Batterien haben wir in einem
Batteriefach untergebracht, das schon einen An/Aus Schalter enthält.

% An example of a floating figure using the graphicx package.
% Note that \label must occur AFTER (or within) \caption.
% For figures, \caption should occur after the \includegraphics.
% Note that IEEEtran v1.7 and later has special internal code that
% is designed to preserve the operation of \label within \caption
% even when the captionsoff option is in effect. However, because
% of issues like this, it may be the safest practice to put all your
% \label just after \caption rather than within \caption{}.
%
% Reminder: the "draftcls" or "draftclsnofoot", not "draft", class
% option should be used if it is desired that the figures are to be
% displayed while in draft mode.
%
\begin{figure}[!t]
  \centering
    \includegraphics[width=3.5in]{images/LED_Platine_Schema_complete.png}
    \caption{Schema der LED Platine}
  \label{fig_led_platine}
\end{figure}

\section{Verwendung}

Sobald man den Bierdeckel anschaltet, muss das leere Glas auf dem Bierdeckel stehen.
Der HX711 wird dann durch den Atmega initialisiert und es wird das Gewicht des leeren Glases
als Offset errechnet, damit man das Gewicht des Glases nicht mitzählt. Wird nun ein volles Glas
abgestellt, wird erkannt dass vom leeren Zustand in den vollen gewechselt wurde, also ein neues
Bier da ist. Dann wird die LED Anzeige eins hochgezählt. Die Anzeige funktioniert,
wie beim klassischen Bierdeckel, mit den Strich/Zaun Zählsystem. Eine 4er Reihe der LEDs
entspricht also den Strichen und eine den Zäunen.

\section{Beschaltung des Atmega328p}

\begin{figure*}[!t]
\centering
\subfloat[Atmega328p Pin Belegung]{\includegraphics[width=4in]{images/atmega328p_pin_belegung.png}%
\label{fig_atmega328_pin}}
\hfil
\subfloat[Konnektoren]{\includegraphics[width=2.5in]{images/bierdeckel_board_connectors.png}%
\label{fig_board_connectors}}
\caption{Beschaltung des Atmega328p}
\label{fig_sim}
\end{figure*}

%\begin{figure}[!t]
%  \centering
%    \includegraphics[width=3.5in]{images/atmega328p_pin_belegung.png}
%    \caption{Atmega328p Pin Belegung}
%  \label{fig_atmega328_pin}
%\end{figure}
%
%\begin{figure}[!t]
%  \centering
%    \includegraphics[width=3.5in]{images/bierdeckel_board_connectors.png}
%    \caption{Konnektoren}
%  \label{fig_board_connectors}
%\end{figure}

Um uns das Leben zu erleichtern haben wir eine Platine entworfen, die den Atmega328p enthält
(siehe Fig. \ref{fig_atmega328_pin}). Die anderen Pins sind mit Konnektoren verbunden (siehe Fig. \ref{fig_board_connectors}).
Wir haben also auf dem Board das den Mikrocontroller enthält mehrere Steckpläzte für die anderen
Komponenten. Das sind also die Steckplätze für:
\begin{itemize}
 \item Das Batteriefach, dass so mit der Platine verbunden wird und die Betriebsspannung (VCC) und die Verbindung zu GND für alle Komponenten liefert.
 \item Den HX711 mit den Pins für VCC, SCK, DT und GND
 \item Die erste Reihe der LEDs auf dem Seperaten LED Board
 \item Die zweite Reihe der LEDs plus einem Pin für GND
 \item Einen ISP Konnektor
 \item die Stecker für die UART Schnittstelle
\end{itemize}

Die letzten beiden tragen nicht zur Funktion unseres Bierdeckels bei aber wurden eingebaut um den
Bierdeckel neu programmieren (über den ISP Konnektor über die serielle Schnittstelle) zu können
falls man das möchte. Über die UART Schnittstelle kann man das Programm dann debuggen.

\section{Funktionsweise des HX711}

\begin{figure*}[!t]
\centering
\includegraphics[width=7in]{images/HX711.jpg}%
\caption{HX711 Beschaltung}
\label{fig_HX711}
\end{figure*}

\begin{figure}[!t]
  \centering
    \includegraphics[width=3.5in]{images/HX711Board.jpeg}
    \caption{HX711 Board}
  \label{fig_HX711_board}
\end{figure}

Der Aufbau der Platine, die den  HX711 IC enthält, ist in Fig. \ref{fig_HX711} dargestellt.
Die Pins PD\_SCK und DOUT (entspricht SCK und DT in Fig. \ref{fig_HX711_board}) werden für den Datenaustausch
verwendet. Sobald DOUT von HIGH auf LOW wechselt, sind Daten zur Abholung bereit.
Dann müssen 25 positive Taktsignale an PD\_SCK geschickt werden (PD\_SCK muss mindestens 2us lang
HIGH und 2us LOW sein, also nicht zu schnell lesen!), wobei jedes ein Bit aus DOUT shifted.
Begonnen wird dabei mit dem MSB Bit. Der 25. Takt zieht DOUT wieder auf HIGH.
Indem weitere Takte gesendet werden (höchstens 27) kann man für das nächste Signal wählen
ob Eingang A oder B ausgelesen wird welchen Gain das gelesene Signal bekommen soll. 

PD\_SCK wird außerdem verwendet um den HX711 resetten. Wenn der Pin länger als 60us auf HIGH
steht dann wird der Power Down Mode aktiviert. Wenn der Pin wieder auf LOW gezogen wird,
dann setzt sich der Chip zurück und geht in den normalen Betriebsmodus über.

\section{Code}

In folgendem Code wird unser HX711 initialisiert indem wir ihn einmal resetten.
Wir setzen also den Pin PD\_SCK (SCLK im Code) 100us lang auf HIGH. Dann lesen wir 32 mal das
Gewicht aus (hx711\_getValue() wird 32 mal aufgerufen) und nehmen den gemittelten Wert als
Offset (entspricht dem Gewicht des leeren Glases,
das zu dem Zeitpunkt auf dem Bierdeckel stehn muss).

\lstinputlisting[language=C]{code/hx711_init.c}

In hx711\_getValue() wird jetzt gewartet bis DOUT (=SDI im Code) auf LOW wechselt.
Dann werden 24 Taktsignale an PD\_SCK (=SCLK im Code) geschickt um die Daten Bit für Bit auszulesen.
Mit dem 25. Takt wird DOUT wieder auf HIGH gesetzt. Da kein weiterer Takt gesendet wurde,
verwenden wir den Input Kanal A des HX711 mit einem Gain von 128 für das nächste Mal.

\lstinputlisting[language=C]{code/hx711_getValue.c}

Die Daten werden dann in mg umgerechnet, mit einem empirisch bestimmten Skalierungsfaktor.

\lstinputlisting[language=C]{code/main.c}

In der main() wird nun in einer Dauerschleife geprüft, ob das abgestellte Glas leer ist oder voll.
hx77\_get\_mg() ruft dabei hx711\_getValue() 32 mal auf um einen gemittelten Wert zu erreichen
und Schwankungen so herauszufiltern. Solange das Glas leer ist blinken die LEDs. Als leer wird das
Glas erkannt (in der Methode isEmpty()), wenn der Inhalt mit einem Gewicht zwischen -50g und 50g gemessen wird.
Als voll wird es erkannt wenn der Inhalt über 480g wiegt (wird in isFull() geprüft). Wird nun erkannt,
dass vom leeren in den vollen Zustand gewechselt wurde, wird daraus geschlossen, dass ein
neues Bier da ist. Dann wird die LED Anzeige eins hochgezählt. Die LEDs werden einzeln über Pins
am Atmega328p angesteuert.

\section{Ausblick}

Zum aktuellen Stand sind die Einzelteile lose miteinander verbunden. Deswegen ist ein Case
geplant, in dem wir unsere Bestandteile unterbringen wollen. Dieses sollte möglichst klein sein
auch wenn es schwer werden dürfte den Durchmesser eines Standard Bierdeckels einzuhalten.\\
Denkbare Erweiterungen wären ein Bluetooth Modul einzubauen und eine App zu erstellen.
Damit könnte man sich die Anzahl der Biere auch auf dem Handy anzeigen lassen und zum Beispiel
einen Rechnter einbauen, der berechnet ob und wann man wieder fahrtauglich ist.


% references section

% can use a bibliography generated by BibTeX as a .bbl file
% BibTeX documentation can be easily obtained at:
% http://www.ctan.org/tex-archive/biblio/bibtex/contrib/doc/
% The IEEEtran BibTeX style support page is at:
% http://www.michaelshell.org/tex/ieeetran/bibtex/
%\bibliographystyle{IEEEtran}
% argument is your BibTeX string definitions and bibliography database(s)
%\bibliography{IEEEabrv,../bib/paper}
%
% <OR> manually copy in the resultant .bbl file
% set second argument of \begin to the number of references
% (used to reserve space for the reference number labels box)


\begin{thebibliography}{1}

\bibitem{hx711sheet}
AVIA Semiconductor, "24-Bit Analog-to-Digital Converter (ADC) for Weigh Scales", HX711 Datasheet
\\
\bibitem{atmelavrsheet}
Atmel, "ATMEL 8-BIT MICROCONTROLLER WITH 4/8/16/32KBYTES", Datasheet, Nov. 2015

\end{thebibliography}


\end{document}


